\pagestyle{fancy}
\section{Βασικά στοιχεία Ουράνιας Μηχανικής}

\subsection{Στοιχεία της Τροχιάς}
Η κίνηση των πλανητών περιγράφεται απο τους τρεις περίφημους  {\bf νόμους του \en  Kepler}, οι οποίοι ερμηνεύουν την γεωμετρία της τροχιάς και την φαινόμενη κίνηση των πλανητών, χωρίς όμως να αποκαλύπτουν το αίτιο της κίνησης. Σύμφωνα με αυτούς:

\begin{enumerate}
\item Οι τροχιές των πλανητών είναι επίπεδες ελλείψεις, με τον Ήλιο να βρίσκεται στη μια εστία.
\item Η κίνηση γύρω απο τον Ήλιο γίνεται με σταθερή εμβαδική ταχύτητα.
\item Τα τετράγωνα των περιόδων περιφοράς των πλανητών, $Τ$, είναι ανάλογα των κύβων των μεγάλων ημιαξόνων, $\alpha$, της τροχιάς τους.

\begin{equation}\label{eq:3rdKeplLaw}
  T^2 = \frac{(4\pi^2){\text{a}}^3}{G(M+m)} \; \footnote{$G$ η Παγκόσμια σταθέρα, $Μ$ η μάζα του ελκτικού κέντρου-στην περίπτωση αυτή του αστέρα-, $m$ η μάζα του πλανήτη}  
\end{equation}
\end{enumerate}

Οι παραπάνω νόμοι διέπουν την κίνηση περιφοράς ενός πλανήτη ή ενός δορυφόρου ή και ενός μικρότερου σώματος γύρω απο το ελκτικό του κέντρο στο διάστημα, καθώς τα σώματα μικρότερης μάζας, $m$, είναι δέσμια απο το βαρυτικό πεδίο του μεγαλύτερου σώματος μάζας $Μ$(ελκτικό κέντρο).\\

Στη συνέχεια της εργασίας θα μιλήσουμε για την κίνηση τέτοιων σωμάτων γύρω απο το ελκτικό τους κέντρο, οπότε οι παραπάνω νόμοι θα μας είναι ιδιαίτερα χρήσιμοι. 

\newpage

\begin{figure}[h]
\centering
 \begin{subfigure}{0.48\textwidth}
   \centering
   \includegraphics[width=\linewidth]{Elipse}
   \caption{Έλλειψη}\label{fig:Elipse}
 \end{subfigure}\hfill
 \begin{subfigure}{0.48\textwidth}
  \centering
  \includegraphics[width=\linewidth]{Elipse2}
  \caption{Ελλειπτική τροχιά Πλανήτη γύρω απο τον Ήλιο}\label{fig:Elipse2}
 \end{subfigure}\hfill
\end{figure}

%\vspace{0.3cm}

Προκειμένου να μελετήσουμε την τροχιά ενός σώματος στο χώρο ή καλύτερα να γνωρίζουμε την ακριβή του θέση σε αυτόν, συναρτήσει του χρόνου, χρειάζεται να εισάγουμε έξι μεταβλητές, μια για κάθε βαθμό ελευθερίας. Για αυτό το λόγο, ορίζουμε τα {\it στοιχείας της τροχιάς}, ένα σύνολο έξι μεταβλητών που χρησιμοποιούνται όχι μόνο για την περιγραφή του σχήματος και τον πλήρη ορισμό του προσανατολισμού της ελλειπτικής τροχιάς αλλα και για τη θέση του σώματος πάνω σε αυτήν.\\

Πιό συγκεκριμένα ορίζουμε:

\renewcommand{\labelenumii}{\roman{enumii}}
\begin{enumerate}

 \item Τα δύο πρώτα στοιχεία σχετίζονται με το σχήμα της έλλειψης.
 
  \begin{enumerate}
  \item Το {\it μεγάλο ημιάξονα} της έλλειψης, $\alpha$, που αποτελεί το άθροισμα του {\it περικέντρου\footnote{Το σημείο μιας ελλεπτικής τροχιάς που βρίσκεται πιο κοντά στην εστία αναφοράς, ευθύγραμμο τμήμα {\bf ΗΠ} Σχήμα:\ref{fig:Elipse2}} ή περιηλίου} και του {\it αποκέντρου\footnote{Το σημείο μιας ελλεπτικής τροχιάς που βρίσκεται πιο μακριά απο την εστία αναφοράς, ευθύγραμμο τμήμα {\bf ΗΠ'} Σχήμα:~\ref{fig:Elipse2}} ή αφηλίου} διαιρούμενο δια δύο, Σχήμα:~\ref{fig:Elipse}.
 
 \item Την {\it εκκεντρότητα} της έλλειψης, $e$, που ορίζεται ως ο λόγος της απόστασης της εστίας απο το κέντρο ($\gamma$) προς τον μεγάλο ημιάξονα της $\alpha$, Σχήμα:~\ref{fig:Elipse}. Η εκκεντρότητα σχετίζεται με το σχήμα της έλλειψης και η τιμή της κυμαίνεται από $0 \leq e < 1$. 'Oσο πλησιάζει στο μηδέν τείνει να γίνει κύκλος άρα και η ελλειπτική τροχιά τείνει να γίνει κυκλική.
 \end{enumerate}

 \item Τα επόμενα τρία στοιχεία σχετίζονται με τον προσανατολισμό της έλλειψης.

 \begin{enumerate}
 \setcounter{enumii}{2}
  \item Την {\it κλίση του επιπέδου της τροχιάς}, $i$ , που ορίζεται ως η γωνία που σχηματίζει το επίπεδο της τροχιάς ενός σώματος με το επίπεδο αναφοράς $x-y$. Ως επίπεδο αναφοράς συνήθως επιλέγεται το επίπεδο της {\it εκλειπτικής}\footnote{Εκλειπτική ονομάζεται το επίπεδο περιφοράς της Γής γύρω απο τον Ήλιο} και οι τιμές του $i$ κυμαίνονται απο $0-180$ μοίρες, Σχήμα:~\ref{fig:KeplerianElements}.
  
  \item Το {\it μήκος του αναβιβάζοντος συνδέσμου}, $\Omega$, το οποίο ορίζεται στο επίπεδο αναφοράς $x-y$ (εκλειπτική) ως η γωνία μεταξύ του άξονα {\en Ox} και της γραμμής των συνδέσμων. Η γραμμή των συνδέσμων αποτελεί την τομή μεταξύ του επιπέδου αναφοράς $x-y$ (εκλειπτική) και της έλλειψης (επίπεδο της τροχίας του σώματος), Σχήμα:~\ref{fig:KeplerianElements}. Ο άξονας {\en Ox} συνήθως επιλέγεται ώστε να συμπίπτει με τη διεύθυνση του μέσου εαρινού σημείου {\it γ}\footnote{Το σημείο της Ουράνιας Σφαίρας στο οποίο φαίνεται να βρίσκεται ο Ήλιος απο τη Γή κατα τη στιγμή της {\it εαρινής ισημερίας} του Βόρειου Ημισφαιρίου της}.
  
  \item Το {\it όρισμα του περιηλίου}, $\omega$, που ορίζεται στο επίπεδο της ελλειπτικής τροχιάς ως η γωνία μεταξύ της γραμμής των συνδέσμων και της θέσης του περιηλίου της τροχιάς, Σχήμα:~\ref{fig:KeplerianElements} .
 \end{enumerate}

 \item Το τελευταίο στοιχείο σχετίζεται με τη θέση του σώματος επι της έλλειψης.
 
 \begin{enumerate}
 \setcounter{enumii}{6} 
   \item Η {\it αληθής ανωμαλία} $\nu$, που ορίζεται στο επίπεδο της ελλειπτικής τροχιάς ως η γωνία μεταξύ της γραμμής των αψίδων και της επιβατικής ακτίνας του σώματος, Σχήμα:~\ref{fig:KeplerianElements} ή αντίστοιχα στο Σχήμα:~\ref{fig:Elipse2}. Η $\nu$ αυξάνει κατα τη φορά κίνησης του σώματος.\\
 Στο σημείο αυτό πρέπει να αναφερθεί ότι αντί της αληθής ανωμαλίας $\nu$ συνήθως χρησιμοποιούμε τη {\it μέση ανωμαλία} {\it Μ}, η οποία έχει διαστάσεις γωνίας και ορίζεται ως $n(t-t_p) = M$ ,όπου
 
 
 \begin{equation}\label{eq:MeanMotion}
  n= \frac{2\pi}{Τ}
 \end{equation}

\vspace{0.3cm}  
  
και ορίζεται ως η μέση (γωνιακή) συχνότητα περιφοράς του σώματος ή {\it μέση κίνηση}. Ουσιαστικά μας δίνει τη θέση ενός ιδεατού κινητού που ακολουθεί ομαλή κυκλική κίνηση σε κύκλο ακτίνας $\alpha$, με συχνότητα περιφοράς ίση προς τη {\it μέση κίνηση} $n$ της πραγματικής τροχιάς.
 \end{enumerate}
\end{enumerate}

\begin{figure}[h]
\en
  \centering
  \includegraphics[scale=1]{KeplerianElements}
  \gr
  \caption{Τα Στοιχεία της Τροχιάς ενός Σώματος}\label{fig:KeplerianElements}
\end{figure}

Στην περίπτωση που η κλίση είναι μηδέν $i=0$ ή αντίστοιχα το επίπεδο της τροχιάς του σώματος ταυτίζεται με το επίπεδο αναφοράς $x-y$ τα στοιχεία $\Omega$ και $ \omega$ βρίσκονται στο ίδιο επίπεδο. Για την αποφυγή αυτής της σύγχυσης ορίζουμε το {\it μήκος του περιηλίου} $\varpi$, το οποίο έχει διαστάσεις γωνίας και δίνεται ως $\varpi = \Omega+\omega$. Εύκολα διακρίνει κανείς ότι στην περίπτωση $ i \neq 0$  η γωνία $\varpi$ αποτελεί μια <<σπαστή>> γωνία σε δύο επίπεδα (απεικονίζεται ως $ω$ για $i=0$ στο Σχήμα:~\ref{fig:Elipse2}, απεικονίζεται για $ i\neq 0$ στο Σχήμα:~\ref{fig:KeplerianElements}).\\
Ακόμα, στην περίπτωση μηδενικής εκκεντρότητας της τροχίας του σώματος $ e=0$ (ισοδύναμα στην περίπτωση κυκλικής τροχιάς) η {\it μέση ανωμαλία} {\it Μ} δεν ορίζεται$\cdot$ καθώς δεν ορίζεται ο {\it χρόνος διάβασης του περικέντρου} $t_p$ και τόσο η εμβαδική όσο και η γραμμική ταχύτητα του σώματος είναι σταθερές. Για την αποφυγή αυτής της σύγχυσης ορίζουμε το {\it μέσο μήκος} $\lambda$, όπου $\lambda= Μ+\varpi$.

Τα έξι στοιχεία της τροχιάς ($ \alpha,e,i,\Omega,\omega,Μ$) βρίσκονται σε αντιστοιχία με τις έξι παραμέτρους που χρησιμοποιούνται στην Μηχανική ($x,y,z,u_x,u_y,u_z$), τις συνιστώσες δηλαδή των διανυσμάτων θέσης και ταχύτητας, ώστε να προσδιορίσουμε με ακρίβεια τις αρχικές συνθήκες της κίνησης.
 
\subsection{Συντονισμοί}

Μελετώντας τώρα τις τροχιές των ουράνιων σωμάτων στο διάστημα μπορούμε να διαπιστώσουμε ότι εμφανίζονται διάφοροι συντονισμοί στην κίνηση τους. Οι συντονισμοί μπορούν να διακριθούν σε διάφορες κατηγορίες, δύο απο αυτές είναι:

\begin{enumerate}

 \item {\it Αιώνιοι Συντονισμοί {\en (secular resonance)}}

 \item {\it Συντονισμοί μέσης κίνησης {\en (mean notion resonance)}}
 
\end{enumerate}

$\rightarrow$ Η πρώτη κατηγορία αναφέρεται σε αυτούς που σχετίζονται με τον συντονισμό της \underline{συχνότητας} {\it μετάπτωσης του περιηλίου, $(g)$}, του {\it μήκους αναβιβάζοντος συνδέσμου, $(s)$}, ή κάποιου γραμμικού συνδυασμού των δύο συχνοτήτων μετάπτωσης. Αυτοί μπορούν να προκαλέσουν μεταβολές στην κλίση και στην εκκεντρότητα της τροχιάς ενός σώματος σε μεγάλες χρονικές κλίμακες.\\

Ένα παράδειγμα για αυτή την κατηγορία αποτελεί ο {\it αιώνιος συντονισμός} $g_H : g_J = 1:1$, όπου η συχνότητα μετάπτωσης του περιηλίου του Ερμή είναι σχεδόν ίδια με τη συχνότητα μετάπτωσης του περιηλίου του Δία.\\

$\rightarrow$ Η δεύτερη κατηγορία συντονισμών αναφέρεται σε αυτούς που σχετίζονται με τις περιόδους κίνησης δύο ή και περισσότερων σωμάτων (για περισσότερα απο δύο σώματα ονομάζονται συντονισμοί {\en Laplace}). Όταν ο λόγος της μέσης κίνησης του ενός σώματος ως προς το άλλο ισούται με ένα λόγο μικρών ακεραίων της μορφής $\frac{p}{p+q}$ , όπου $p$ και $q$ ακέραιοι, τότε τα σώματα βρίσκονται σε συντονισμό τάξης $q$. {\it Τέτοιου είδους συντονισμοί μπορούν είτε να αποσταθεροποίησουν είτε να σταθεροποιήσουν την τροχιά ενός σώματος ανάλογα με το μέγεθος τους αλλά και με το πόσο κοντά θα οδηγηθούν τα σώματα σε αυτόν (ακριβής τιμή του συντονισμού) ή όχι. Προφανώς αν οδηγηθούν σε ακριβή τιμή του συντονισμού η επιρροή του - είτε ως προς αστάθεια είτε ως προς σταθεροποίηση του συστήματος- μεγιστοποιείται}.\\


Καθώς ένα σώμα περιφέρεται γύρω απο το ελκτικό του κέντρο η μέση κίνηση του σώματος είναι αντιστρόφως ανάλογη της περιόδου περιφοράς,\eqref{eq:MeanMotion}. Άρα μπορούμε να πούμε ισοδύναμα ότι όταν ο λόγος της περιόδου περιφοράς του ενός σώματος, ως προς του άλλου, ισούται με ένα λόγο μικρών ακέραιων της μορφής $\frac{p}{p+q}$, τότε τα σώματα βρίσκονται σε συντονισμό τάξης $q$.\\


   \underline{{\bf H γεωμετρία ενός συντονισμού μέσης κίνησης}}
\vspace{0.3cm}

Για να γίνει πιο κατανοητή {\it η γεωμετρία ενός συντονισμού μέσης κίνησης} θα γίνει μια μικρή ανάλυση μέσω ενός απλοϊκού συλλογισμού.\\

Ας υποθέσουμε ότι ένας αστεροειδής της κύριας ζώνης είναι σε συντονισμό $(p+q:p)=2:1$, τάξης $q=1$ με τον Δία. Για λόγους απλότητας θεωρούμε ότι η τροχιά του Δία είναι κυκλική, συνεπίπεδη με αυτή του αστεροειδή και αγνούμε τις διαταραχές (ή παρέλξεις) που προκαλεί η μεταξύ τους βαρυτική αλληλεπίδραση. Έστω τώρα ότι την χρονική στιγμή $t=0 \; sec$ τα δύο σώματα βρίσκονται σε σύνοδο, στο περίκεντρο της τροχιάς του αστεροειδή. Τότε τη χρονική στιγμή $t= \frac{1}{2} T_J$ ο αστεροειδής έχει συμπληρώσει μια πλήρη περιφορά γύρω απο τον Ήλιο και έχει επιστρέψει στην αρχική του θέση ενώ ο Δίας μισή περιφορά και βρίσκεται στην διεύθυνση του απογείου της τροχιάς του αστεροειδή. Την χρονική στιγμή $t=  T_J$ ο Δίας έχει επιστρέψει στην αρχική του θέση έχοντας συμπληρώσει μια πλήρη περιφορά και ο αστεροειδής βρίσκεται εκ νέου στην αρχική του θέση έχοντας συμπληρώσει άλλη μια πλήρη περιφορά (συνολικά δύο πλήρεις περιφορές). Τελικά μετά απο χρόνο $t=T_J$ η αρχική διάταξη επαναλαμβάνεται.\\
Ουσιαστικά αν ο αστεροειδής της κύριας ζώνης είναι σε συντονισμό $p+q:p$ με τον Δία, η αρχική διάταξη επαναλαμβάνεται κάθε $p+q$ πλήρεις περιφορές του αστεροειδή γύρω απο τον Ήλιο.\\

  \underline{{\bf Συντονισμοί και λόγος μεγάλων ημιαξόνων των τροχιών}}
\vspace{0.3cm}

Ας θεωρήσουμε δύο σώματα με μάζες $m$, $m'$, με μεγάλους ημιάξονες $\alpha, \alpha'$ και με συνεπίπεδες τροχιές, τα οποία περιφέρονται γύρω απο κοινό ελκτικό κέντρο μάζας $Μ$. Επίσης ας αγνοήσουμε τις διαταραχές που προκαλεί η μεταξύ τους βαρυτική αλληλεπίδραση. Η μάζα $m$ αναφέρεται στο εσωτερικό σώμα ενώ η $m'$ στο εξωτερικό , δηλαδή $ \alpha < \alpha'$. Τότε η συνθήκη συντονισμού για τα δύο σώματα είναι: 

\begin{equation}\label{eq:ResonanceCondition1} 
 \frac{n'}{n}=\frac{p}{p+q}, \;  \text{όπου $p$ και $q$ ακέραιοι} \cite[{\en Chap.~8, Sect.~8.2}]{murray1999solar}
\end{equation}

Μέσω της σχέση \eqref{eq:3rdKeplLaw} προκύπτει:

\begin{equation}
 (\frac{T}{T'})^2=(\frac{\alpha}{\alpha'})^3 \frac{M+m'}{M+m}   
\end{equation}

Ας πάμε ένα βήμα παραπέρα και ας υποθέσουμε ότι τα δύο σώματα αναφέρονται σε δύο πλανήτες του Ηλιακού μας Συστήματος (έστω τον Δία και τον Κρόνο) και ότι το κεντρικό σώμα αντιστοιχεί στον ίδιο τον Ήλιο. Τότε καθώς ισχύει $m<<M$ και $m'<<M$ συνεπάγεται ότι:

\begin{equation}\label{eq:Ratio}
 \frac{\alpha}{\alpha'}=(\frac{T}{T'})^{2/3}   
\end{equation} 

και μέσω της \eqref{eq:MeanMotion} μπορεί να εκφραστεί ώς: 

\begin{equation}\label{eq:RatioMeanMotion}
 \frac{\alpha}{\alpha'}=(\frac{n'}{n})^{2/3}   
\end{equation}

Ο συνδυασμός των \eqref{eq:ResonanceCondition1} και \eqref{eq:RatioMeanMotion}:

\begin{equation}\label{eq:ResonanceCondition2}
\frac{p}{p+q}=(\frac{\alpha}{\alpha'})^{3/2}, \; \text{όπου $p$ και $q$ ακέραιοι}   
\end{equation}

μας δείχνει ότι {\it η συνθήκη συντονισμού} για τα δύο σώματα μπορεί να εκφραστεί και μέσω του λόγου των μεγάλων ημιαξόνων της τροχιάς τους, ο οποίος αν ικανοποιεί την \eqref{eq:ResonanceCondition2} τα σώματα βρίσκονται σε συντονισμό $p+q:p$, τάξης $q$.\\
Για δεδομένες τιμές της περιόδου περιφοράς των δύο πλανητών και του μεγάλου ημιάξονα της τροχίας του Δία η λύση της \eqref{eq:ResonanceCondition2} ως προς $\alpha'$ ουσιαστικά μας δίνει την τιμή που θα έπρεπε έχει ο μεγάλος ημιάξονας του Κρόνου ώστε να ισχυεί συντονισμός.\\

Όπως αναφέραμε η ύπαρξη συντονισμών {\it μέσης κίνησης} μπορούν είτε να {\it αποσταθεροποίησουν} είτε να {\it σταθεροποιήσουν} την τροχιά ενός σώματος.\\

Ένα παράδειγμα αποσταθεροποίησης της τροχιάς σωμάτων αποτελούν οι συντονισμοί 4:1, 3:1, 5:2, 7:3 και 2:1 των αστεροειδών της κύριας ζώνης με τον Δία. Αστεροειδείς της ζώνης με μεγάλους ημιάξονες $\alpha$ που εμπίπτουν στους παραπάνω συντονισμούς έχουν ασταθείς τροχιές με αποτέλεσμα να διαφεύγουν απο τη ζώνη. Το φαινόμενο αυτό είναι τόσο έντονο ώστε η αριθμητική πυκνότητα αστεροειδών συναρτήσει της απόστασης $\alpha$ εμφανίζει απότομα ελάχιστα που αγγίζουν μηδενικές τιμές πυκνότητας, Σχήμα:~\ref{fig:KirkwoodGaps}. Το φαινόμενο παρατήρησε πρώτος ο {\en Kirkwood (1867)}, για αυτό και ονομάζονται {\it διάκενα} του {\en Kirkwood (Kirkwood gaps)}.

\begin{figure}[h!]
  \centering
  \includegraphics[scale=0.48]{KirkwoodGaps}
  \gr
  \caption{Αριθμητική Κατανομή Αστεροειδών της Κύριας Ζώνης}\label{fig:KirkwoodGaps}
\end{figure}

\newpage

Ένα παράδειγμα σταθεροποίησης της τροχίας σωμάτων αποτελούν οι συντονισμοί τριών εκ των τεσσάρων Γαλιλαϊκων δορυφόρων του Δία. Πιο συγκεκριμένα οι δορυφόροι Γανυμήδης, Ευρώπη και Ιό σχηματίζουν {\it τριπλό συντονισμό} ή {\it συντονισμό} {\en {\it Laplace}}, με αποτέλεσμα οι τροχίες τους να έχουν λόγο περιόδων 1:2:4. Αυτό σημαίνει ότι για κάθε μια πλήρη περιφορά του Γανυμήδη γύρω απο τον Δία η Ευρώπη εκτελεί δύο και η Ιό τέσσερις, Σχήμα:~\ref{fig:LaplaceResonance}.

\begin{figure}[h!]
  \centering
  \includegraphics[scale=0.33]{LaplaceResonance}
  \gr
  \caption{Τριπλος Συντονισμός 1:2:4 μεταξυ των τριών εκ των τεσσάρων Γαλιλαϊκων δορυφόρων του Δία }\label{fig:LaplaceResonance}
\end{figure}

Ενώ αρχικά οι παλιρροιογόνες δυνάμεις του Δία προς αυτούς αλλα και η μεταξύ τους βαρυτική αλληλεπίδραση μετέβαλλαν τους μεγάλους ημιάξονες $\alpha$  των τροχιών τους με διαφορετικό ρυθμό και ως εκ τουτο ο λόγος των περιόδων περιφοράς άλλαζε με συνεχή τρόπο πιθανόν να βρίσκονταν μακριά απο κάποιο συντονισμό και με την πάροδο του χρόνου να βρεθηκαν σε συντονισμό {\en({\it resonance trapping})}. Ξέρουμε όμως\cite[{\en Chap.~1, Sect.~1.8}]{tausigmaiotagammaacutealphanuetavarsigma2015pilambdaalphanuetatauiotakappaacutealpha}, ότι τελικά το σύστημα <<κλειδώθηκε>> σε αυτό το συντονισμό και πια οι όποιες μεταβολές των τροχιών λόγω παλίρροιας, γίνονται με τέτοιον τρόπο ώστε ο {\it λόγος των περιόδων να διατηρείται σταθερός}.
 


\newpage
\section{Πεδίο Ακτινοβολίας}

{\it Πεδίο ακτινοβολίας} ονομάζεται ο χώρος όπου υπάρχει ηλεκτρομαγνητική ακτινοβολία. Αυτή μπορεί να εκπέμπεται, να απορροφάται ή και απλός να διαδίδεται στο χώρο αυτό. Απο τον παραπάνω όρισμο καταλαβαίνουμε ότι η έννοια του πεδίου ακτινοβολίας καλύπτει μεγάλο εύρος περιπτώσεων, ώστε ένας τέτοιος χώρος να μπορεί να θεωρηθεί η ατμόσφαιρα ενός πλανήτη, ένα μεσοαστρικό μέσο, ένα αστέρι κ.ο.κ.

 \subsection{Βασικά Μεγέθη Διάδοσης Ακτινοβολίας}
 
Η κατανομή θερμοκρασιών σε ένα δίσκο σκόνης εξαρτάται έντονα απο την ακτινοβολία του αστέρα, ο οποίος περιβάλλεται απο τον δίσκο. Η ακτινοβολία αρχικά εκπέμπεται απο τον αστέρα, διαδίδεται, απορροφάται απο τα σωματίδια της σκόνης θερμαίνοντας τον δίσκο και στη συνέχεια μέρος της ακτινοβολίας επανεκπέται απο αυτά. Προκειμένου να μελετήσουμε την παραπάνω διαδικασία πρέπει να ορίσουμε συγκεκριμένα μεγέθη.

\begin{itemize}
 
 \item {\it Ροή της Ακτινοβολίας}\\
  
Ας υποθέσουμε μια στοιχειώδη επιφάνεια $dA$, η ενέργεια ανα μονάδα επιφάνειας και ανα μονάδα χρόνου ορίζει την {\it ροή της ακτινοβολίας} που περνάει απο αυτή την επιφάνεια:

\begin{equation}\label{eq:Flux}
  F=\frac{dE}{dAdt}, \; \frac{erg}{sec \; cm^2}
\end{equation}

Φυσικά η ενέργεια αυτή είναι ηλεκτρομαγνητικής φύσης, άρα είναι σημαντικο το {\it μήκος κύματος} ή αντίστοιχα η {\it συχνότητα} της ακτινοβολίας ώστε να προσδιορίσουμε το ποσό της ενέργειας που περνάει απο την επιφάνεια. Οι σχέσεις συνδέονται με τη γνωστή σχέση $E=h \nu $ ή αντίστοιχα $E=\frac{h}{\lambda}$, όπου $h$ η σταθερά του {\en Planck}.\\

Η σχέση \eqref{eq:Flux} ορίστηκε για την ενέργεια σε όλα τα μήκη κύματος και ονομάζεται {\it βολομετρική} ροή, συνήθως όμως μας ενδιαφέρει η ροή της ακτινοβολίας σε ένα συγκεκριμένο εύρος μηκών κύματος (ή αντίστοιχα συχνοτήτων). Έτσι ορίζουμε την {\it μονοχρωματική} ροή:

\begin{equation}\label{eq:FluxBolometric}
  F_{\nu} =\frac{dE}{dAdtd\nu}, \; \frac{erg}{sec \; cm^2 Hz}  
\end{equation}

 και προφανώς οι δύο ποσότητες συνδέονται μέσω των σχέσεων:
 
 \begin{equation}\label{eq:FluxBolMon}
  F= \int_{0}^{\infty} F_{\lambda} d\lambda \; , \; F= \int_{0}^{\infty} F_{\nu} d\nu
\end{equation}
 
Στο σημείο αυτό ορίζουμε ως μονάδα μέτρησης της μονοχρωματικής ροής, $F_{\lambda}$, το \underline{{\en Jansky}}, όπου $1 Jy= 10^{-23} erg \; sec^{-1} \; cm^{-2} \; Hz^{-1}$\\

\item {\it Ειδική Ένταση της Ακτινοβολίας}\\

Η ροή της ακτινοβολίας αναφέρεται στην συνολική ενέργεια που μεταφέρουν όλα τα φωτόνια, \underline{απο όλες τις διευθύνσεις}, καθώς διασχίζουν μια επιφάνεια. Όταν μελετάμε την διάδοση της ακτινοβολίας μας ενδιαφέρει {\it η ενέργεια που μεταφέρεται σε μια συγκεκριμένη διεύθυνση}. Έτσι χρησιμοποιούμε την {\it ειδική ένταση της ακτινοβολίας}, η οποία αποτελεί το βασικό μέγεθος της θεωρίας διάδοσης ηλεκτρομαγνητικών κυμάτων και ορίζεται ως: 

\begin{equation}\label{eq:SpecificIntensity}
  I_{\nu}=\frac{dE_{\nu}}{\cos{\theta} dAdtd\nu d\Omega} , \; \frac{erg}{sec \; cm^2 \; Hz \; ster}
\end{equation}

\begin{figure}[h]
\en
  \centering
  \includegraphics[scale=0.5]{SpecificIntensity}
  \gr
  \caption{Ορισμός της Ειδικής Έντασης της Ακτινοβολίας}\label{fig:SpecificIntensity}
\end{figure}


Έστω πάλι η στοιχειώδης επιφάνεια $dA$ απο την οποία περνούν ακτίνες, με διάνυσμα κατέυθυνσης $\vec{n}$, οι οποίες εμπεριέχονται σε μια στοιχειώδη στερεά γωνία $d\Omega$. Η ενέργεια αυτή που περνάει απο την επιφάνεια $dA$ (και μεταφέρεται απο τις ακτίνες εντός της στερέας γωνίας $d\Omega$) σε ένα συγκεκριμένο έυρος συχνοτήτων, $d\nu$, ανα μονάδα χρόνου, $dt$, ορίζεται απο την ειδική έντασης της ακτινοβολίας. Η γωνία $\theta$ είναι η γωνία μεταξύ του κάθετου διανύσματος στην στοιχειώδη επιφάνεια ${dA}$ και του μοναδιαίου διανύσματος $\vec{n}$, που ορίζει την στερέα γωνία μέσα στην οπoία διαδίδεται η ακτινοβολία.\\

Απο τις \eqref{eq:FluxBolometric} και \eqref{eq:SpecificIntensity} παρατηρούμε ότι τα δύο μεγέθη συνδέονται έτσι ώστε:

\begin{equation}\label{eq:FluxBolSpecInte}
  {F_{\nu}} = \int_{\Omega_s} I_{\nu}\cos{\theta} d\Omega  
\end{equation}

\item {\it Μέση 'Ενταση της Ακτινοβολίας}\\

Η {\it μέση ένταση της ακτινοβολίας} είναι η μέση τιμή της ειδικής έντασης της ακτινοβολίας, ολοκληρωμένη σε στερεά γωνία $d\Omega$, δηλαδή:

\begin{equation}\label{eq:MeanIntensity}
  J_{\nu} = \frac{1}{4\pi} \int_{4\pi} I_{\nu}(\vec{n}) d\Omega  
\end{equation}

{\it Το μέγεθος αυτό έχει ιδιαίτερο ενδιαφέρον, διότι όπως θα δούμε παρακάτω  καθορίζει την θέρμανση των στερεών σωματιδίων της σκόνης.}
 \end{itemize}
 
\subsection{Ιδιότητες Μελανού Σώματος}

Ένα υλικό, θερμοκρασίας $T$, εκπέμπει ακτινοβολία. Όταν η ύλη και η ακτινοβολία έρθουν σε {\it Θερμοδυναμική Ισορροπία}, η οποία επιτυγχάνεται αφού ένα φωτόνιo αλληλεπιδράσει πολλές φορές με την ύλη (οπτικά αδιαφανές μέσo), τότε η ειδική ένταση της ακτινοβολίας του δίνεται απο την συνάρτηση {\en Planck}.  

\begin{equation}\label{eq:SpecInteBlackBody}
  I_{\nu} = B_{\nu}(T) = \frac{2h{\nu}^3}{c^2} \frac{1}{e^{\frac{h\nu}{kT}}-1}
\end{equation}

H οποία φυσικά μπορεί να εκφραστεί και σαν συνάρτηση του μήκους κύματος

\begin{equation}\label{eq:SpecInteBlackBody2}
  I_{\lambda} = B_{\lambda}(T) = \frac{2hc^2}{\lambda^5} \frac{1}{e^{\frac{hc}{\lambda kT}}-1}
\end{equation}

όπου η θερμοκρασία μπορεί να είναι και συνάρτηση της θέσης μέσα στο υλικό $Τ=Τ(r)$.\\

Το πεδίο ακτινοβολίας μελανού σώματος είναι ένα ομοιόμορφο και ισοτροπικό πεδίο ακτινοβολίας σε θερμοδυναμική ισορροπία και μέσω των σχέσεων \eqref{eq:MeanIntensity} και \eqref{eq:SpecInteBlackBody} η μέση ένταση της ακτινοβολίας του είναι:

\begin{equation}\label{eq:MeanInteBlackBody1}
  J_{\nu} = \frac{1}{4\pi}\int_{4\pi} I_{\nu} d\Omega = \frac{I_{\nu}}{4\pi}\int_{4\pi} d\Omega = I_{\nu} = \frac{1}{4\pi} \frac{2h{\nu}^3}{c^2} \frac{1}{e^{\frac{h\nu}{kT}}-1} \int_{4\pi} d\Omega = B_{\nu}(T)
\end{equation}

Απο τις σχέσεις \eqref{eq:FluxBolSpecInte} και \eqref{eq:SpecInteBlackBody} ολοκληρώνοντας για σφαιρικές συντεταγμένες στη μισή σφαίρα:

\begin{equation}\label{eq:PossitiveFlux}
  {F_{\nu}}^{+} = I_{\nu}\int_{0}^{2\pi} \int_{0}^{\frac{\pi}{2}} \cos{\theta} \sin{\theta}  d\theta d\phi = \pi I_{\nu} =  \pi \frac{2h{\nu}^3}{c^2} \frac{1}{e^{\frac{h\nu}{kT}}-1}
\end{equation}

Ακόμα οι ολικές ποσότητες προκύπτουν εύκολα αν ολοκληρώσουμε σε όλο το ηλεκτρομαγνητικό φάσμα. Η ολική ένταση της ακτινοβολίας που προέρχεται απο μια σφαιρική επιφάνεια είναι: 

\begin{equation}\label{eq:TotalInteBlackBody}
  Ι = \int_{0}^{\infty} I_{\nu} d\nu = \int_{0}^{\infty} B_{\nu}(T) d\nu = \frac{\sigma T^4}{\pi} 
\end{equation}

όπου $\sigma = \frac{2{\pi}^5k^4}{15c^2h^3}$ η σταθερά του {\en Stefan-Boltzmann}.\\ 

Απο τις σχέσεις \eqref{eq:PossitiveFlux} \eqref{eq:TotalInteBlackBody} ολική θετική ροή:

\begin{equation}\label{eq:TotalPossitiveFlux}
  F^{+} = \int_{0}^{\infty} {F_{\nu}}^{+} d\nu = \sigma T^4
\end{equation}

Μια πολύ σημαντική ιδιότητα του μελανού σώματος είναι η εξής:\\
{\it Το μήκος κύματος, $\lambda_{max}$, στο οποίο η εκπομπή γίνεται μέγιστη εξαρτάται μόνο απο την θερμοκρασία, $T$.}

\begin{equation}\label{eq:WienLaw}
  \lambda_{max} = \frac{0.2898}{T} \; cm
\end{equation}

Η παραπάνω σχέση ορίζει ορίζεται ως {\it Νόμος του {\en Wien}} και προκύπτει αν θέσουμε ίση με το μηδέν την πρώτη παράγωγο της \eqref{eq:SpecInteBlackBody2} ως προς το μήκος κύματος $\lambda$ . Φυσικά ο νόμος του {\en Wien} μπορεί να εκφραστεί αντίστοιχα και συναρτήση της συχνότητας.

\subsection{Διάδοση Ακτινοβολίας}

Μια απο τις θεμελειώδεις ιδιότητες της {\it ειδικής έντασης} της ακτινοβολίας, $I_{\nu}$ είναι ότι \underline{η τιμή της δε μεταβάλλεται με την απόσταση} στον κενό χώρο του διαστήματος. Έστω ότι ακολουθούμε μια ακτίνα που διαδίδεται στον κενό χώρο και διανύει απόσταση $s$, τότε αποδεικνύεται ότι:

\begin{equation}\label{eq:ConstantSpecInte}
  \frac{dI_{\nu}}{ds}=0 
\end{equation}

Η ιδιότητα αυτή είναι υψίστης σημασίας και παίζει κεντρικό ρόλο στη διάδοση της ακτινοβολίας.\\

Καθώς μια ακτίνα διαδίδεται στο χώρο μεταφέρει ενέργεια $Ε$. Όταν αυτή περνάει μέσα απο ύλη, πυκνότητας $\rho$, ενέργεια $dE$ μπορεί ειτε να προστεθεί είτε να αφαιρεθεί μέσω {\it εκπομπής} ή {\it απορρόφησης} αντίστοιχα$\cdot$ με αποτέλεσμα η ειδική ένταση της ακτινοβολίας, $I_{\nu}$, να μεταβάλλεται.\\

Η ενέργεια που εκπέμπεται απο έναν στοιχειώδη όγκο, $dV$, καθορίζεται απο τον {\it συντελεστή αυθόρμητης εκπομπής}, $j_{\nu}$:

\begin{equation}\label{eq:Emissivity}
  dE_{\nu} = j_{\nu}\rho dVdtd\nu d\Omega , \; \text{όπου $j_{\nu}$ έχει διαστάσεις} \; \frac{erg}{sec \; g \; Hz \; ster}  
\end{equation}

Aν λάβουμε υπόψην ότι $dV=dAds$ και απο την σχέση \eqref{eq:Emissivity} τότε η ενέργεια που προστίθεται καθώς η ακτίνα ταξιδεύει απόσταση $ds$ μέσα στην ύλη είναι:

\begin{equation}\label{eq:Emissivity2}
  \frac{dI_{\nu}}{ds}=j_\nu \rho
\end{equation}

Αντίστοιχα, καθώς η ακτίνα ταξιδεύει απόσταση $ds$ μέσα στην ύλη, πυκνότητας $\rho$, η ενέργεια που χάνεται λόγο απορρόφησης καθορίζεται απο τον {\it συντελεστή απορρόφησης}, $k_{\nu}$, ο οποίος ορίζει την {\it αδιαφάνεια} του υλικού:

\begin{equation}\label{eq:Absorption1}
  \frac{dI_{\nu}}{ds}=-k_{\nu}\rho I_\nu, \; \text{όπου $k_{\nu}$ σε $\frac{cm^2}{g}$}
\end{equation}

Ο συντελεστής απορρόφησης με τη σειρά του εξαρτάται απο τα χαρακτηριστικά της ύλης. Πιο συγκεκριμένα ας υποθέσουμε ένα σωματίδιo της ύλης, ο {\it συντελεστής απορρόφησης} του εξαρτάται απο {\it το σχήμα, το μέγεθος και την σύσταση} του σωματιδίου. Σαν αποτέλεσμα, αν υποθέσουμε ότι τα σωματίδια έχουν σφαιρικό σχήμα, ίδιο μέγεθος και ίδια σύσταση, ο αριθμός των σωματιδίων που θα συναντήσει η ακτίνα επιφάνειας $dA$ ταξιδεύοντας απόσταση $ds$ μέσα στην ύλη εξαρτάται απο την αριθμητική πυκνότητα του υλικου, $n$ (σωματίδια ανα μονάδα όγκου). Υποθέτοντας ότι κάθε σωματίδιο έχει ενεργό διατομή\footnote{Ουσιαστικά η ενεργός διατομή αποτελεί μια υποθετική δραστική επιφάνεια γύρω από έναν απορροφητή, μέσα στην οποία κάποιο εισερχόμενο φωτόνιo μπορεί να απορροφηθεί} απορρόφησης $\sigma_{\nu}$  ο συντελεστής απορρόφησης δίνεται:

\begin{equation}\label{eq:Absorption2}
 k_{\nu} = \frac{n \sigma_{\nu}}{\rho}, \; \frac{cm^{2}}{g}
\end{equation}

Είναι προφανές ότι το γινόμενο $k_{\nu}\rho$ εκφράζει την πιθανότητα απορρόφησης ενός φωτονίου που διασχίζει μήκος $ds$ εντός του υλικού. Έτσι η {\it μέση ελεύθερη διαδρομή}\footnote{Η μέση ελεύθερη διαδρομή ενός φωτονίου ορίζεται ως η απόσταση που διασχίζει το φωτόνιο πρωτού απορροφηθεί}, $l$, είναι το μήκος

\begin{equation}\label{eq:MeanFreePath}
 l_{\nu} = \frac{1}{k_{\nu}\rho} 
\end{equation}

Απο τις σχέσεις \eqref{eq:Emissivity2} και \eqref{eq:Absorption1} προκύπτει η {\it εξίσωση διάδοσης της ακτινοβολίας}:

\begin{equation}\label{eq:RadiationTransferEquation}
  \frac{dI_{\nu}(s)}{ds}=j_{\nu}(s)\rho - k_{\nu}\rho I_{\nu}(s) 
\end{equation}

Η παραπάνω σχέση είναι ιδιαίτερα σημαντική καθώς μας δίνει την μεταβολή της ειδικής έντασης της ακτινοβολίας καθώς αυτή διασχίζει απόσταση $s$ εντός ύλης.\\

Μια εξίσου σημαντική ποσότητα είναι το {\it οπτικό βάθος}, $\tau_{\nu}$. Το οπτικό βάθος ενός μέσου ορίζεται ως ο αριθμός των {\it μέσων ελεύθερων διαδρομων} ενός φωτονίου μέχρι αυτό να απορροφηθεί. Το οπτικό βάθος μεταξύ δύο σημείων $s_1$ και $s_2$ ορίζεται ώς:

\begin{equation}\label{eq:OpticalDepth}
  \tau_{\nu}(s_1,s_2) = \int_{s_1}^{s_2} k_{\nu}(s)\rho ds
\end{equation}

Έτσι σε μια δεδομένη συχνότητα, $\nu$, ένα μέσο λέγεται ότι έχει μεγάλο οπτικό βάθος ({\en optically thick medium}) αν $\tau_{\nu} > 1$ ένω ότι έχει μικρό οπτικό βάθος ({\en optically thin medium}) αν $\tau_{\nu} < 1$.

Ένα ακόμα σημαντικό μέγεθος είναι η ποσότητα $S_\nu$, που ονομάζεται {\it συνάρτηση της πηγης} και εκφράζει την εκπεμπτικότητα του υλικού.

\begin{equation}\label{eq:SourceFunction}
 S_\nu = \frac{j_\nu}{k_\nu}
\end{equation}

Απο τις \eqref{eq:RadiationTransferEquation} και \eqref{eq:SourceFunction} προκύπτει ότι :

\begin{equation}\label{eq:RadiationTransferEquation2}
  \frac{dI_{\nu}(\tau_{\nu})}{d\tau_{\nu}}= S_{\nu}(\tau_{\nu}) - I_{\nu}(\tau_{\nu})\\
\end{equation}

Η εξίσωση διάδοσης της ακτινοβολίας, \eqref{eq:RadiationTransferEquation2}, είναι γραμμική εξίσωση πρώτης τάξης και η γενική λύση της εξίσωσης δίνεται με άθροιση της λύση της ομογενούς με μια μερική λύση της πλήρους Δ.Ε.. Αρχικά προσδιορίζουμε την λύση της ομογενούς Δ.Ε., η οποία είναι $I_{\nu}(\tau_{\nu}) = c e^{-\tau_{\nu}}$, όπου $c = I_{\nu}(\tau_{\nu}=0)$. Μια μερική λύση της πλήρους Δ.Ε. προκύπτει με τη {\it μέθοδο της μεταβολής αυθαίρετων σταθερών} ή {\it μέθοδο {\en Lagrange}} θεωρώντας λύση της μορφής $I_{\nu}(\tau_{\nu}) = z(\tau_{\nu}) e^{-\tau_{\nu}}$, όπου $z(\tau_{\nu})$ μια συνάρτηση την οποία θα προσδιορίσουμε. Απο την \eqref{eq:RadiationTransferEquation2} προκύπτει:

\begin{align*}
\frac{dz(\tau_{\nu})}{d\tau_{\nu}} = S_{\nu}(\tau_{\nu}) e^{\tau_{\nu}}\\
z(\tau_{\nu}) = c + \int_{0}^{\tau_{\nu}} S_{\nu}(\tau_{\nu})e^{\tau_{\nu}} d\tau_{\nu}\\
I_{\nu}(\tau_{\nu}) = I_{\nu}(\tau_{\nu}=0) + \int_{0}^{\tau_{\nu}} S_{\nu}(\tau_{\nu})e^{\tau_{\nu}} d\tau_{\nu}\\ 
\end{align*}

και αν η συνάρτηση της πηγής είναι ανεξάρτητη της θέσης μέσα στο υλικό ή αντίστοιχα οι συντελεστές απορρόφησης και εκπομπής είναι σταθεροί για όλο το υλικό τότε η λύση της Δ.Ε. μπορεί να γραφεί ως:

\begin{equation}\label{eq:RadiationTransferEquation3}
I_{\nu}(\tau_{\nu}) = I_{\nu}(0)e^{-\tau_{\nu}} + S_{\nu}(1-e^{-\tau_{\nu}})
\end{equation}

θεωρώντας αμελητέα την ακτινοβολία πίσω απο το στρώμα του υλικού, ({\en negligible background intensity}), συνεπάγεται ότι:

\begin{equation}\label{eq:RadiationTransferEquation4}
I_{\nu}(\tau_{\nu}) =  S_{\nu}(1-e^{-\tau_{\nu}})
\end{equation}

Η σημασία της {\it συνάρτησης της πηγής} φανερώνεται αν υποθέσουμε ότι το υλικό στο οποίο διαδίδεται η ακτινοβολία βρισκεται σε {\it Θερμοδυναμική Ισορροπία}. Σε αυτή την περίπτωση η ακτινοβολία του δίνεται παντού απο την συνάρτηση {\en Planck} και κατεπέκταση $\frac{dI_{\nu}}{ds}=0$. Απο την σχέση \eqref{eq:RadiationTransferEquation2} προκύπτει:

\begin{equation}\label{eq:RadiationTransferEquationBlackBody}
  S_{\nu} = I_{\nu} = B_{\nu}(T)\\
\end{equation} 

Η σχέση \eqref{eq:RadiationTransferEquationBlackBody} μας λέει ότι όταν ένα υλικό βρίσκεται σε θερμοδυναμική ισορροπία {\it η ένταση της ακτινοβολίας δεν μεταβάλλεται στο εσωτερικό του, δηλαδή όση ακτινοβολία απορροφάται σε καθε σημείο του εσωτερικού του, τόση επανεκπέμπεται}!

\subsection{Αλληλεπίδραση Μεταξύ Ακτινοβολίας και Σωματιδίων της Σκόνης}

Στην περίπτωση διάδοσης της ακτινοβολίας σε έναν πρωτοπλανητικό δίσκο σκόνης, εξετάζουμε την αλληλεπίδραση της με τα στερεά σωματίδια της σκόνης. Προκειμένου να αναλύσουμε τις παραμέτρους του προβλήματος είναι απαραίτητο να εμβαθύνουμε περισσότερο στην φυσική σημασία του συντελεστή απορρόφησης και να εξηγήσουμε το πως καθορίζει το ποσό της ενέργειας που απορροφούν και το ποσό που εκμπέμπουν τα σωματίδια. Για την απλοποίηση του προβλήματος θεωρούμε ότι τα σωματίδια είναι ομογενείς σφαίρες.\\ 

Στην εξίσωση \eqref{eq:Absorption2} δώσαμε τον ορισμό του συντελεστή απορρόφησης για {\bf δεδομένο μήκος κύματος παρατήρησης}, $\lambda$, ή αντίστοιχα δεδομένη συχνότητα, $\nu$. Σαν αποτέλεσμα ανάλογα με το μέγεθος του σωματιδίου και το μήκος κύματος παρατήρησης η τιμή του συντελεστή απορρόφησης μεταβάλλεται. Πιο συγκεκριμένα ανάλλογα με την παράμετρο μεγέθους $x=\frac{2 \pi r_{d}}{\lambda}$.\\

Η εξάρτηση αυτή φανερώνεται μέσα απο την ενεργό διατομή απορρόφησης, η οποία είναι:

\begin{equation}\label{eq:CrossSection}
 \sigma_{\nu} = Q_{\nu} Α
\end{equation}

όπου $Α$ η γεωμετρική επιφάνεια που συμμετέχει στην διαδικασία απορρόφησης.\\

Ο συντελεστής $Q_{\nu}$ ονομάζεται {\it αποδοτικότητα απορρόφησης} ({\en absorption efficiency}), είναι ένα αδιάσταστο μέγεθος και εξαρτάται απο την παράμετρο μεγέθους $x=\frac{2 \pi r_{d}}{\lambda}$. Ο υπολογισμός της $Q_{\nu}$ δίνεται απο την θεωρία σκέδασης του {\en Mie}, η $Q_{\nu}$ καθορίζει το πόσο <<καλά>> απορροφά την ακτινοβολία, συχνότητας $\nu$, ένα σφαιρικό σωματίδιο ακτίνας $r_{d}$. Για διαφορετική παράμετρο μεγέθους $x=\frac{2 \pi r_{d}}{\lambda}$ παίρνουμε και διαφορετική λύση της $Q_{\nu}$. Γενικά είναι ένας περίπλοκος υπολογισμός που απαιτεί την χρήση υπολογιστικής ισχύς$\cdot$ πάραυτα μπορούμε να δώσουμε κάποιες λύσεις για δύο διαφορετικά όρια.

\begin{itemize}
 \centering
  \item $x \geq 0.1 \Longrightarrow \frac{2 \pi r_{d}}{\lambda} \geq 0.1$
  \item $x \leq 0.1 \Longrightarrow \frac{2 \pi r_{d}}{\lambda} \leq 0.1$
\end{itemize}

Το πρωτο όριο ονομάζεται {\it Όριο Γεωμετρικής Οπτικής}. Ουσιαστικά όταν το μήκος κύματος είναι αρκετά μικρότερο απο το μέγεθος του σωματιδίου, τότε αναδύεται η σωματιδιακή φύση του φωτός και το σωματίδιο απορροφάει όλο το φώς που πέφτει στην επιφάνεια του. Άρα $Q_{\nu}=1$.\\

Το δεύτερο όριο ονομάζεται {\it Όριο {\en Rayleigh}}. Ουσιαστικά όταν το μήκος κύματος είναι συγκρίσιμο ή μεγαλύτερο απο το μέγεθος του σωματιδίου, τότε αναδύεται η κυματική φύση του φωτός και η αποδοτικότητα $Q_{\nu}$ εξαρτάται απο την συχνότητα $\nu$. Αποδεικνύεται ότι, \cite[{\en Chap.~8, Sect.~2.2}]{krugel2002physics} στο όριο {\en Rayleigh}, $x=\frac{2 \pi r_{d}}{\lambda} \ll 1$ για σφαιρικά σωματίδια, συγκεκριμένης σύστασης, η λύση μπορεί να βρεθεί αναλυτικά και ο συντελεστής $Q_{\nu}$ είναι ανάλογος της παραμέτρου $x=\frac{2 \pi r_{d}}{\lambda}$ και μιας συνάρτησης που εξαρτάται μόνο απο την συχνότητα $\nu$. Η συνάρτηση μπορεί να εκφραστεί σε εκθετική μορφή και τελικά:

\begin{equation}\label{eq:AbsorEfficiency}
\centering 
Q_{\nu}=r_{d}Q_{0}{\nu}^{\beta}, \; \text{όπου $Q_{0}$ σταθερά}  
\end{equation}  

Το $\beta$ είναι ένας σταθερός αριθμός που σχετίζεται με το μέγεθος, το σχήμα και την σύσταση των σωματιδίων της σκόνης. Οι αλλαγές στην τιμή του $\beta$ συνδέονται με αλλαγές στις ιδιότητες των σωματιδίων της σκόνης. Συνήθως για την μεσοαστρική σκόνη ({\en ISM Dust}) $1.7 \leq \beta_{ISM} \leq 2$ και αποδεικνύεται ότι η μείωση της τιμής του $\beta$ συνεπάγεται αύξηση στο μέγεθος των σωματιδίων \cite{draine2006submillimeter}. Ετσι για πρωτοπλανητικούς δίσκους μεγάλης ηλικίας, όπου αναμένουμε σωματίδια μεγαλύτερων διαστάσεων αναμένουμε και μικρότερες τιμές του $\beta\cdot$ αντίθετα για πρωτοπλανητικούς δίσκους νεαρής ηλικίας, όπου αναμένουμε σωματίδια μικρότερων διαστάσεων αναμένουμε και μεγαλύτερες τιμές του $\beta$.\\

Έτσι η τιμή του του συντελεστή απορρόφησης $k_{\nu}$ μπορεί να δωθεί για τα δύο όρια ώς:

\begin{align}
k_{\nu} = \frac{Α}{m},\; \text{για} \; x \geq 0.1 \nonumber\\
k_{\nu} = \frac{Α}{m} r_{d}Q_{0}{\nu}^{\beta},\; \text{για} \; x \leq 0.1 \label{eq:Absorption3a}  \\  
\end{align}

όπου $m=\frac{4 \pi \rho r_d^3 }{3}$ η μάζα του σωματιδίου για σφαιρικά σωματίδια.\\

Τελικά ο συντελεστής απορρόφησης συναρτήσει του μήκους κύματος για ένα σωματίδο δίνεται ως:

\begin{equation}\label{eq:Absorption4}
 k_{\lambda} = k_{0} (\frac{{\lambda}}{\lambda_{0}})^{-\beta}
\end{equation}

όπου οι σταθερές $k_0$ και $\lambda_{0}$ προσδιορίζονται απο την \eqref{eq:Absorption2} για ένα σωματίδιο μάζας $m$ και ακτίνας $r_d$. Πιο συγκεκριμένα γνωρίζοντας την τιμή της αποδοτικότητας για το συγκεκριμένο μήκος κύματος, $\lambda_{0}$, μπορούμε να προσδιορίσουμε το $k_0$.

\newpage
\section{Πρωτοπλανητικοί Δίσκοι}

  \subsection{Δημιουργία και Γενικά Χαρακτηριστικά}  
  
  Όταν η μάζα του αρχικού νέφους, αερίου και σκόνης, ξεπεράσει μια κρίσιμη τιμή μάζας που ονομάζεται \textbf{Μάζα \en Jeans}\footnote{H κρίσιμη τιμή μάζας του νέφους, η οποία αν ξεπεραστεί ξεκινάει η βαρυτική κατάρρευση του} ακολουθεί το φαινόμενο της {\it βαρυτικής κατάρρευσης}\footnote{Όταν η βαρυτική δύναμη της μάζα των νεφών ξεπεράσει τη δύναμη που προκαλείται απο την βαθμίδα της πίεσης τους ακολουθεί η συρρίκνωση τους, ένα φαινόμενο που στην αστρονομική ορολογία ονομάζεται {\it βαρυτική κατάρρευση}}. Καθώς λαμβάνει χώρα η συρρίκνωση, το νέφος περιστρέφεται με ολοένα και μεγαλύτερη γωνιακή ταχύτητα, $ω$, ώστε να παραμείνει σταθερή η στροφορμή του, $J=Μ \omega R^2$ με αποτέλεσμα το νέφος να αποκτά πεπλατισμένη δισκοειδή μορφή. Διακρίνοντας την εξέλιξη του συστήματος σε δύο στάδια έχουμε:  τον πρωτοαστρικό δίσκο και στη συνέχεια τον {\it πρωτοπλανητικό δίσκο}\footnote{Στα περισσότερα μοντέλα εξέλιξης πλανητικών συστημάτων κάνουμε λόγο και για τρίτο στάδιο όπου έχουμε έναν {\en debris disk} που προκύπτει απο την σύγκρουση πλανειτοειδών κατά τα τελικά στάδια δημιουργίας των πλανητών}. Κατα τη μετάβαση απο το πρώτο στάδιο στο δεύτερο δημιουργείται μια μεγάλη συγκέντρωση δεσμεύοντας τεράστια ποσά αερίου και σκόνης στο κέντρο του δίσκου , το {\it πρωταστέρι}, το οποίο στο τέλος αυτής έχει εξελιχθεί στον αστέρα του συστήματος$\cdot$ έτσι ο πρωτοπλανητικός δίσκος χαρακτηρίζεται απο μικρότερη μάζα σε σχέση με τον πρωτοαστρικό δίσκο. Στον υπόλοιπο δίσκο δημιουργούνται μικρότερες συγκεντρώσεις ύλης, οι οποίες ακολουθούν (σε μεγάλη προσέγγιση) ελλειπτικές τροχιές περί τον αστέρα. Τα σωματίδια σκόνης του δίσκου απορροφούν ενέργεια μέσω της αστρικής ακτινοβολίας και στη συνέχεια την επανεκπέμπουν στα {\en far-IR} και {\en submillimeter} μήκη κυματος. Η ακτινοβολία σε αυτές τις υψηλές ράδιο συχνότητες είναι και ο λόγος που κάνουμε χρήση ραδιοτηλεσκοπείων σαν την {\en ALMA} για να χαρτογραφήσουμε τέτοιους δίσκους. <<Καρπός>> των παρατηρήσεων πρωτοπλανητικών δίσκων και της θεωρίας της στατιστικής είναι το συμπέρασμα ότι: 
  
  
\begin{enumerate}
  \item \underline{Μάζα} \\  
   Η μάζα τους κυμαίνεται απο περίπου $0.001$ έως $0.1$ Μ$\odot$\footnote{Μ$\odot$ είναι ο συμβολισμός που χρησιμοποιείται για την Μάζα του Ήλιου, 1Μ$\odot$= 1 Ηλιακή Μάζα καθώς το σύμβολο $\odot$ αναφέρεται στον Ήλιο} και οι διάμετροι  τους απο $30$ έως και $200$ {\en AU}\footnote{Η τιμή του μεγάλου ημιάξονα της τροχίας της Γης, όπως προκύπτει απο τον 3ο Νόμο του {\en Kepler}, για περίοδο ίση με 1 έτος  365.25 ημέρες, ${\en AU}=1$ Αστρονομική Μονάδα}. 
   \item \underline{Χρόνος Ζωής} \\ 
  Είναι αρκετά βραχύβιοι σε σχέση με την κοσμική κλίμακα του χρόνου. Πιο συγκεκριμένα ο {\it μέσος χρόνος ζωής} τους είναι περίπου {\en $3Myr$}, με τους μεγαλύτερους που έχουν παρατηρηθεί να έχουν ηλικία περίπου {\en $10Myr$}.   
\end{enumerate} 
   
  \subsection{Φυσικά Χαρακτηριστικά και Δομή του Δίσκου}
  
Οι πρωτοπλανητικοί δίσκοι παρουσιάζουν {\it αξονική} και {\it κατοπτρική συμμετρία}. Ακόμα μπορούμε να κάνουμε την προσέγγιση απειροστού πάχους καθώς η ακτινική διεύθυνση τους, $r$, είναι τάξεις μεγέθους μεγαλύτερη απο την κατακόρυφη διέθυνση $z$, δηλαδή:
  
  \begin{equation}\label{eq:ZeroWidth}
    \frac{z}{r} << 1
  \end{equation}
  
   Καθώς κινούμαστε ακτινικά σε μεγαλύτερες αποστάσεις, απο τον αστέρα προς τα πέρατα του δίσκου, ο λόγος του πάχους του δίσκου, $H$, προς την απόσταση, $r$, αυξάνεται. Οι δίσκοι αυτοί ονομάζονται {\en{\it Flaring Discs}}, όλα τα σημεία της επιφάνειας τους δέχονται απευθείας την ακτινοβολία του αστέρα, θερμαίνονται και στη συνέχεια επανεκπέμπουν μέρος της ακτινοβολίας σε μεγάλες αποστάσεις.\\
      
Στην περίπτωση που το πάχος του δίσκου, $H$, είναι σταθέρο σε κάθε τιμή της απόστασης, $r$, κάνουμε λόγο για {\en {\it Flat Discs}}, οι οποίοι και θα μας απασχολήσουν στην παρούσα εργασία.\\

Οι πρωτοαστρικοί και κατεπέκταση και οι πρωτοπλανητικοί δίσκοι χαρακτηρίζονται απο προφίλ {\it επιφανειακής πυκνότητας}, $\Sigma(r)$. Σύμφωνα με την\textbf{\en{ MMSN} {\it Minimum Mass Solar Nebula})}\cite[{\en Chap.~7, Sect.~7.2}]{tausigmaiotagammaacutealphanuetavarsigma2015pilambdaalphanuetatauiotakappaacutealpha}, μπορούμε να κάνουμε μια εκτίμηση για το προφίλ επιφανειακής πυκνότητας του πρωτοαστρικού δίσκου του Ηλιακού μας Συστήματος. Το αποτέλεσμα είναι επιφανειακή πυκνότητα μορφής:

 \begin{equation}\label{eq:SurfaceProfile}
 \Sigma(r) = \Sigma_0(\frac{r}{1AU})^{-\gamma} \; \frac{g}{cm^2}  
 \end{equation}

όπου για στερεά σώματα σκόνης $\Sigma_0=7$ για $ r < 2.7 AU$ ή $\Sigma_0=30$ για $ r > 2.7 AU$ και $\gamma=\frac{3}{2}$. Βλέπουμε ότι η επιφανειακή πυκνότητα της σκόνης μείωνεται εκθετικά καθώς μεγαλώνει η απόσταση απο το κέντρο του δίσκου ανάλογα με τον όρο $\frac{1}{\gamma}$. \\

Βάση της υπάρχουσας θεωρίας, πιστεύουμε ότι το Ηλιακό μας Συστήματος δημιουργήθηκε με παρόμοιο τρόπο με άλλα Πλανητικά Συστήματα. Σαν αποτέλεσμα αυτού αλλα και του γεγονότος ότι ο νόμος της βαρύτητας δεν ισχύει μόνο τοπικά στο Ηλιακό μας Σύστημα αναμένουμε παρόμοιας μορφής καμπύλη επιφανειακής πυκνότητας και για άλλους πρωτοπλανητικούς δίσκους.\\
   
Η θερμοκρασία ενός πρωτοπλανητικού δίσκου εξαρτάται απο διάφορους παράγοντες. Άρχικα το αστέρι στο κέντρο του δίσκου θερμαίνει τον δίσκο με την ακτινοβολία του, καθώς τα στερεά σωματίδια σκόνης σε ορισμένα μήκη κύματος απορροφούν\footnote{Τα σωματίδια σκόνης μπορούν επίσης να σκεδάσουν ακόμα και πολώσουν την ακτινοβολία, αλλά οι μηχανισμοί αυτοί δε θα μας απασχολήσουν στην παρούσα εργασία} ενώ σε μεγαλύτερα μήκη κύματος την επανεκπέμπουν. Το ποσό της ακτινοβολίας που θα απορροφηθεί εξαρτάται τόσο απο τη γεωμετρία του δίσκου όσο και απο τις ιδιότητες των σωματιδίων της σκόνης.
Προκειμένου να υπολογίσουμε την κατανομή θερμοκρασιών στον δίσκο και κατ' πέκταση να εξάγουμε το θερμοκρασιακό του προφίλ πρέπει να συνυπολογίσουμε τους μηχανισμούς θέρμανσης και ψύξης του δίσκου.\\   

Στην περίπτωση του επίπεδου πρωτοπλανητικού δίσκου σκόνης \underline{(χωρίς αερίο)} {\it ο βασικός μηχανισμός θέρμανσης είναι η απορρόφηση της ακτινοβολίας του αστέρα που περιβάλλει}. 
Φυσικά η κατανομή θερμοκρασίας που θα αποκτήσει ο δίσκος σκόνης εξαρτάται απο διάφορους παράγοντες, οι κυριότεροι όμως είναι:
 
   \begin{enumerate}
      \item H βολομετρική ροή, $F\odot$, του αστέρα
      \item O συντελεστής απορρόφησης, $k_\nu$, των σωματιδίων της σκόνης
      \item Η επιφανειακή πυκνότητα μάζας του δίσκου     
   \end{enumerate}
   
\subsection{Σκόνη των Πρωτοπλανητικών Δίσκων}

Στην ορολογία της Αστροφυσικής, η ύλη που εντοπίζεται μεταξύ των άστρων ονομάζεται {\en Interstellar Medium, ISM}. Στον χώρο αυτό εντοπίζουμε ύλη, ηλεκτρομαγνητική ακτινοβολία, βαρυτικό και μαγνητικό πεδίο. Η μεσοαστρική ύλη αποτελείται απο $70\%$ υδρογόνο, $28\%$ ήλιο και $2\%$ βαρύτερα στοιχεία όπως οξυγόνο, άνθρακα, άζωτο κ.ο.κ. Το $99\%$ της ύλης βρίσκεται στην αέρια φάση, ({\en gas}), ενώ μόλις περίπου το $1\%$ βρίσκεται στην στερεά κατάσταση, ({\en solid state}), το οποίο ονομάζουμε {\it σκόνη}. Σαν αποτέλεσμα η συνολική μάζα της σκόνης είναι τάξεις μεγέθους μικρότερη απο την συνολική μάζα του αερίου.\\
Η μεσοαστρική σκόνη αποτελείται απο πολύ μικρά στερεά σωματίδια, το μέγεθος των οποίων εκτείνεται απο $0.0005$μ$m$ έως περίπου $1$μ$m$ \cite[{\en Chap.~1, Sect.~2.5}]{tielens2005physics} και απαρτίζεται κυρίως απο πυριτικά άλατα, ενώσεις άνθρακα και πιθανότατα γραφίτη. Ακόμα είναι {\it αδιαφανής} στο ορατό κομμάτι του φάσματος, ενώ εκπέμπει σε διαφορετικά μήκη κύματος ανάλογα με το σχήμα, το μέγεθος και την σύσταση των σωματιδίων που την αποτελούν.
Τα σωματίδια που έχουν μεγάλο μέγεθος, $\geq 0.01$ μ$m$, βρίσκονται σε θερμοδυναμική ισορροπία, θερμοκρασίας $T_{dust}$, με το εκάστοτε πεδίο ακτινοβολίας στο οποίο περιέχονται. Έτσι η ακτινοβολία που εκπέμπουν, σύμφωνα με τον νόμο του {\en Kirchhoff}, χαρακτηρίζεται αποκλειστικά απο την θερμοκρασία ισορροπίας τους, $T_{dust}$.\\

Η μεσο αστρική ύλη που εντοπίζεται γύρω απο τους νεαρούς αστέρες αποτελεί όπως είδαμε τους πρωτοπλανητικούς δίσκους. Σε αυτό το στάδιο, τα σωματίδια της σκόνης μπορούν να αυξηθούν σε μέγεθος κατα αρκετές τάξεις μεγέθους, απο μ$m$ μέχρι τον σχηματισμό στερεών πλανητών. Σε συνδυασμό με το όριο ηλικίας αυτών των δίσκων αναμένουμε ότι ποιοτικά όσο μεγαλύτερος σε ηλικία είναι ένας δίσκος τόσο πιθανότερο είναι να συνατήσουμε σε αυτόν σωματίδια μεγαλύτερων διαστάσεων αλλα και να μην συναντήσουμε αέριο. Φυσικά η απώλεια του αερίου στον δίσκο και η αύξηση του μεγέθους των σωματιδίων δεν εξαρτάται μόνο απο τον χρόνο αλλά από μία πληθώρα παραγόντων που επικρατούν στον δίσκο καθώς το πρόβλημα του σχηματισμού συσσωμάτων και τελικά πλανειοτειδών είναι ένα εξαιρετικά περίπλοκο πρόβλημα. Απο την άλλη πλευρά, ποιοτικά περιμένουμε ότι όσο πιο <<νεαρός>> είναι ένας δίσκος τόσο πιθανότερο είναι να συνατήσουμε σε αυτόν σωματίδια μικρότερων διαστάσεων αλλά και μεγαλύτερες ποσότητες αερίου. 
 

\section{Αριθμητική Προσομοίωση}
\en
  \subsection{Solar System Integration Software Package- SWIFT}

\gr
Για την αριθμητική ολοκλήρωση των τροχιών των σωματιδίων, χρησιμοποιήθηκε ο συμπλεκτικός αλγόριθμος ολοκλήρωσης {\en {\it SWIFT}\cite{levison2013swift}(A Solar System Integration Software Package)}, ο οποίος έχει σχεδιαστεί ώστε να ενσωματώνει ένα σύνολο αμοιβαία βαρυτικά αλληλεπιδρώντων σωμάτων μαζί με μια ομάδα σωματιδίων {\en (test particles)}. {\bf Τα {\en test particles} <<αισθάνονται>> τη  βαρυτική δύναμη των μαζικών σωμάτων αλλα δεν αλληλεπιδρούν βαρυτικά μεταξύ τους ούτε επηρεάζουν τα μαζικά σώματα}. Ο {\en {\it SWIFT}} είναι ικανός να πραγματοποιήσει {\it ολοκληρώσεις μεγάλης κλίμακας {\en (long-term integrations)}} χαρτογραφόντας, με αρκετά μικρό βήμα χαρτογράφησης ({\en timestep}), τον {\it χώρο φάσης}\footnote{Ο χώρος φάσης είναι ένας χώρος στον οποίο εκπροσωπούνται όλες οι πιθανές καταστάσεις ενός συστήματος, με κάθε πιθανή κατάσταση να αντιστοιχεί σε ένα μοναδικό σημείο στο χώρο φάσης. Για μηχανικά συστήματα, ο χώρος φάσης αποτελείται συνήθως από όλες τις πιθανές τιμές των μεταβλητών θέσης και ορμής.} του συστήματος σύμφωνα με την μέθοδο {\en ``Wisdom-Holman Mapping``\cite{wisdom1991symplectic} (WHP)}.

Η χρήση του \en {\it SWIFT} \gr για προσομοιώσεις δυναμικής εξέλιξης συστημάτων Ν-σωμάτων έχει αρκετά πλεονεκτήματα. Αρχικά η μέθοδος ολοκλήρωσης και χαρτογράφησης που χρησιμοποιεί, ενώ βασίζεται στη μέθοδο {\en "Wisdom"}\cite{wisdom1982origin}, αποτελεί τη βελτίωμένη εκδοχή της, χωρίς να υπόκειται στους περιορισμούς της αρχικής μεθόδου. Πιο συγκεκριμένα:

 \begin{itemize}
    \item Η εγκυρότητα της χρήσης του αλγορίθμου δεν περιορίζεται μόνο σε χαμηλές τιμές της εκκεντρότητας και της κλίσης των σωμάτων.
    \item Η εγκυρότητα της χρήσης του αλγορίθμου δεν περιορίζεται σε συγκεκριμένες τιμές συντονισμών
 \end{itemize}

Ένα ακόμα πλεονέκτημα είναι ότι ενώ {\bf και} η αρχική μέθοδος ολοκλήρωσης και χαρτογράφησης βασίζεται στη μηχανική Χάμιλτον, η βελτιωμένη εκδοχή της χειρίζεται την εκλογή της <<Χαμιλτονιανής>> του συστήματος με μεγαλύτερη κομψότητα διατηρώντας βέβαια τον συμπλεκτικό χαρακτήρα του αλγορίθμου, δηλαδή διατηρείται ο δισδιάστατος χώρος των φάσεων $dq \wedge dp$, όπου $(q,p)$ οι κανονικές μεταβλητές. Πιο συγκεκριμένα η Χαμιλτονιανή χωρίζεται σε δύο μέρη:


\begin{equation}\label{eq:Hamiltonian1}
  H = H_{Kepler} + {\phi}(t)*H_{Interaction}
\end{equation}

Ο όρος $H_{Kepler}$ αναπαριστά την αλληλεπίδραση του κάθε σώματος με το κεντρικό σώμα (πρωτεύον) του συστήματος (το αστέρι), ο $H_{Interaction}$  αναπαριστά τη \underline{διαταραχή} του καθε δευτερευόν σωμάτος({\en test particles} και πλανήτες) απο την αλληλεπίδραση του με τους πλανήτες και $\phi${\en (t)} μια περιοδική συνάρτηση που εκλέγεται με σεβασμό στη {\it διαδικασία του μέσου όρου}.\\

Η βασική ιδέα πίσω απο την εκλογή της $\phi${\en (t)} είναι ότι οι όροι που διαταράσσονται ταχέα (ή, ισοδύναμα, που προκαλούν διαταραχές υψηλής συχνότητας) έχουν μέσο όρο μηδέν μέσα σε ένα χρονικό διάστημα που η τάξη μεγέθους του είναι ανάλογη της τάξης μεγέθους των περιόδων περιφοράς των σωμάτων του συστήματος. Έτσι η $\phi${\en (t)} μπορεί να εκλεγεί αυθαίρετα αρκεί ο μέσος όρος της σε μια περίοδο χαρτογράφησης να ισούται με {\bf ένα}, ώστε η \eqref{eq:Hamiltonian1} να δίνει, ανα {\en timestep}, την πραγματική Χαμιλτονιανή του συστήματος:

\begin{equation}\label{eq:Hamiltonian2}
  H = H_{Kepler} + H_{Interaction}
\end{equation}

Αυτή η εκλογή της Χαμιλτονιανής, \eqref{eq:Hamiltonian1}, καθιστά πολύ αποτελεσματική την μέθοδο χαρτογράφησης καθώς τα δύο μέρη της μπορούν να {\it ολοκληρωθούν και να υπολογιστούν ανεξάρτητα}.\\

Ένα ακόμα αξιοσημείωτο χαρακτηριστικό του {\en {\it SWIFT}} είναι ο τρόπος που διαχειρίζεται τις κοντινές διελεύσεις μεταξύ {\en test particles} και πλανητών, καθώς χρησιμοποεί μεταβλητό βήμα ολοκλήρωσης (μεταβλητό {\en timestep}). Πιο συγκεριμένα, η ολοκλήρωση τώρα βασίζεται στη μέθοδο {\en ``Regularized Mixed Variable Symplectic``\cite{levison1994long} {\it (RMVS)} method``}, ορίζοντας δύο ζώνες γύρω απο κάθε πλανήτη.
Η εσωτερική ζώνη, αποτελεί μια σφαίρα, με ακτίνα ίση με μια {\it ακτίνα {\en Hill}}\footnote{Η ακτίνα {\en Hill} εκφράζει την σφαίρα επιρροής ενός σώματος, δηλαδή την έκταση της γειτονικής του περιοχής στην οποία η δυναμική καθορίζεται απο την βαρύτητα του σώματος και όχι του αστέρα} και η εξωτερική, που αποτελεί ένα σφαιρικό κέλυφος, εκτείνεται από $r_{H}<r\leq3r_{H}$.\\

\begin{equation}\label{eq:HillRadius}
    r_{H} = {\en a}_{P}(\frac{1}{3}\frac{M_P}{M_{\odot}+M_P})^{\frac{1}{3}}, \; \text{ο δείκτης {\en P} αντιστοιχεί στο "{\en Planet}"}
\end{equation}

Την χρονική στιγμή που ξεκινάει η προσομοίωση αν κάποιο {\en test particle} βρίσκεται εντός της εξωτερικής ζώνης ή προβλέπεται ότι θα βρεθεί εντός αυτής κατά το τέλος του βήματος ολοκλήρωσης (εντός ενός {\en timestep}), τότε το βήμα ολοκλήρωσης μειώνεται κατα ένα παράγοντα 100, δηλαδή

\begin{equation}
  \frac{t_\nu}{t_{\nu+1}} = 100 
\end{equation}

Αντίστοιχα για την εσωτερική ζώνη ο παράγοντας μείωσης είναι 5.
 
\begin{equation}
  \frac{t_\nu}{t_{\nu+1}} = 5  
\end{equation}

Απο τον ορισμό της {\it ακτίνας {\en Hill}} είναι προφανές ότι αν ένα σώμα βρεθεί στην εσωτερική ζώνη, το πρωτεύον σώμα στην $H_{Kepler}$ της τροχιάς του γίνεται ο πλανήτης, έναντι του αστέρα.\\ 
 
{\it Με αυτό τον τρόπο πετυχαίνεται καλύτερη ανάλυση ή αλλιώς παρέχεται μεγαλύτερη διακριτική ικανότητα στον προσδιορισμό των διανυσμάτων θέσεως και ορμής των σωμάτων στην περίπτωση της κοντινής μεταξύ τους διεύλευσης. Αυτή η διαδικασία προφανώς και <<σέβεται>> τη συμπλεκτικότητα, δηλαδή σε κάθε βήμα ολοκλήρωσης ο μετασχηματισμός της Χαμιλτονιανής είναι ένας {\bf κανονικός μετασχηματισμός}}\footnote{Ένας μετασχηματισμός {\en (q,p)→(q$'$,p$'$)} είναι κανονικός (συμπλεκτικός) εάν διατηρεί την μορφή των εξισώσεων του {\en Hamilton}}.\\ 

Στον \en {\it SWIFT} \gr εμπεριέχονται συνολικά οι 4 μέθοδοι ολοκλήρωσης:

\begin{itemize}
 \item O αλγόριθμος \en ``Wisdom-Holman Mapping``\cite{wisdom1991symplectic} {\it (WHP)}.  \href{https://ui.adsabs.harvard.edu/abs/1991AJ....102.1528W/abstract}{WHP}.
 \gr
 \item O αλγόριθμος \en ``Regularized Mixed Variable Symplectic``\cite{levison1994long} {\it (RMVS)} method``,   
    \href{https://ui.adsabs.harvard.edu/abs/1994Icar..108...18L/abstract}{RMVS}.
 \gr
 \item O αλγόριθμος \en ``A fourth order T+U Symplectic (TU4) method``\cite{gladman1991symplectic},  \href{https://ui.adsabs.harvard.edu/abs/1991CeMDA..52..221G?high=38a5d886f206113&db_key=AST}{TU4}.
 \gr
 \item O αλγόριθμος \en ``A Bulirsch-Stoer method``. 
\end{itemize}

Για περισσότερες πληροφορίες επισκεφτείτε \href{http://ascl.net/1303.001}{[{\en SWIFT}]}

\newpage
\section{Παρατηρήσεις σε Υψηλές Ραδιοσυχνότητες}

\subsection{Συνέλιξη}

Η {\it Συνέλιξη} αποτελεί μια μαθηματική πράξη μεταξύ δύο συναρτήσεων ($f$ και $g$) και παράγει μια τρίτη ως εξής:

\begin{equation}
[f\ast g](x) = \int_{- \infty}^{+ \infty} f(x-x')g(x') dx'
\end{equation}\label{Convolution}

Η διαδικασία της συνέλιξης, στην περίπτωση αστρονομικών παρατηρήσεων, είναι πολύ σημαντική. Διότι αν $g(x') \rightarrow I_{\nu}(\vec{n}')$ και $f(x-x') \rightarrow f(\vec{n}-\vec{n}')$ τότε $I^{m}_{\nu}= \iint_{- \infty}^{+ \infty} f(\vec{n}-\vec{n}')I_{\nu}(\vec{n}') dn'$ είναι η ειδική ένταση που μετράει το τηλεσκόπειο ή ειδική ένταση που μετράει η διάταξη των τηλεσκοπείων μας (αν πρόκειται για συμβολομετρία), όπου η $f(\vec{n}-\vec{n}')$ είναι η δέσμη. Τελικά:

\begin{equation}\label{beamRadio}
f(\vec{n}-\vec{n}') = \vec{P_n} (\vec{n}-\vec{n}')
\end{equation}
όπου $\vec{P_n} (\vec{n}-\vec{n}')$ η απόκριση σημειακής πηγής του συστήματος που επιβάλλεται απο τους νόμους της κυματικής οπτικής (λέγεται και {\en point spread function}). Ενώ

\begin{equation}\label{beamOptical}
f(\vec{n}-\vec{n'}) = \vec{S_n} (\vec{n}-\vec{n}')
\end{equation}
όπου $\vec{S_n} (\vec{n}-\vec{n}')$ η συνάρτηση που μας λέει την απόκριση της σημειακής πηγής όπως την καθορίζει η ατμόσφαιρα.\\

Στην περίπτωση της {\it Ραδιοαστρονομίας} η $f(\vec{n}-\vec{n}')$ είναι η απόκριση σημειακής πηγής $\vec{P_n} (\vec{n}-\vec{n}')$.


\en
\subsection{Atacama Large Millimeter Array-ALMA}

\gr

Η {\en ALMA} είναι ένα συμβολόμετρο το οποίο αποτελείται απο 66 ραδιο αντέννες, βρίσκεται στην βόρεια Χιλή και ανιχνεύει ηλεκτρομαγνητική ακτινοβολία σε υψηλές ραδιοσυχνότητες. Πιο συγκεκριμένα το εύρος μηκών κύματος που καλύπτει είναι από $0.3$ έως $3.6mm$ ή αντίστοιχα από $84$ έως $950 GHz$. Έχει κατασκευαστεί σε υψόμετρο $5000m$ στην πεδιάδα {\en Chajnantor} των Χιλιανών Άνδεων, μία τοποθεσία που προσφέρει συνήθως τις εξαιρετικά ξηρές και καθαρές συνθήκες του ουρανού που απαιτούνται για παρατήρησεις στο παραπάνω τμήμα του φάσματος. Ο μεγάλος αριθμός αντέννων δίνει την δυνατότητα διαφορετικών διάταξεων πετυχαίνοντας πολύ υψηλές τιμές διακριτικής ικανότητας ({\en resolution}) και κατ' επέκταση εικόνες υψηλής ευκρίνειας. Σαν αποτέλεσμα είναι το πιό ισχυρό επίγειο τηλεσκόπιο για την παρατήρηση του ψυχρού Σύμπαντος (μοριακό αερίο και σκόνη) και κατ' επέκταση μελετά τα δομικά στοιχεία των αστεριών, των πλανητικών συστημάτων και των γαλαξιών.\\

Όταν δύο κεραίες στοχεύουν ένα αντικείμενο στον ουρανό, η παρατήρηση γίνεται απο ελαφρώς διαφορετικές θέσεις. Σαν αποτέλεσμα η ηλεκρομαγνητική ακτινοβολία από το αντικείμενο φτάνει στη μία κεραία με διαφορά φάσης απο ότι στην άλλη. Ο πολλαπλασιασμός και ο υπολογισμός του μέσου όρου των αντίστοιχων σημάτων μέσω του διασυσχετιστή μας δίνει το πλάτος και την φάση του διασυσχετισμένου σήματος. Ουσιαστικά μετράμε μια ποσότητα που ονομάζεται {\en complex visibility}, $V(u,v)$, η οποία δεν είναι τίποτε άλλο απο τον μετασχηματισμό {\en Fourier} της κατανομής λαμπρότητας της πηγής στο {\it χώρο των χωρικών συχνότητων}, $(u,v) plane$\footnote{Ο χώρος των χωρικών συχνοτήτων σε αναλογία με τον χώρο των φάσεων της Μηχανικής, αποτελεί ένα χώρο όπου εκπροσωπούνται οι δυνατές καταστάσεις του οπτικού συστήματος. Κάθε οπτικό ερέθισμα μπορεί να αναπαρασταθεί απο την ένταση του φωτός με μια κλίμακα χρώματος. Έτσι μια φωτογραφία είναι η αναπαράσταση μιας διασδιάστατης συναρτησης $f(u,v)$ με χωρικές συντεταγμένες $u,v$ που περιγράφει πως κατανέμεται η τιμή της έντασης του φωτός στο επίπεδο της φωτογραφίας} και περιέχει πληροφορίες σχετικά με την λαμπρότητα της πηγής αλλα και με την θέση της σε αυτόν μέσω των χωρικών συντεταγμένων $(u,v)$, δηλαδή:

\begin{equation}\label{ComplexVisibility}
V(u,v) = \iint_{\Omega_s} I(l,m)e^{-2\pi i(ul+vm)} dldm= Ae^{i\phi}
\end{equation}

όπου $A$ το πλάτος και $\phi$ η φάση του διασυσχετισμένου σήματος. Έτσι ο αντίστροφος μετασχηματισμός {\en Fourier} της {\en complex visibility} είναι η κατανομή λαμπρότητας της πηγής στο επίπεδο παρατήρησης, $I_{\nu}(l,m)$! Άρα:

\begin{align}
V(u,v) = \iint_{\Omega_s} I_{\nu}(l,m)e^{-2\pi i(ul+vm)} dldm\\
I_{\nu}(l,m) = \iint_{\Omega_s} V(u,v)e^{2\pi i(ul+vm)} dudv
\end{align}

όπου

\begin{equation}
(ul+vm) = \vec{D_{\lambda}} \vec{\sigma}
\end{equation}


και $\vec{\sigma}=(l,m)$ είναι το διάνυσμα θέσης ενός σημείου μέσα στην πηγή σχετικά με το κέντρο του συμβολομετρικού χάρτη. Ακόμα $\vec{D_{\lambda}} = (u,v)$ που είναι οι προβολές του $\vec{D_{\lambda}}$ πάνω στους άξονες του ουρανογραφικού συστήματος συντεταγμένων της πηγής. Οι $Ν=66$ ραδιο αντέννες του {\en ALMA} μετράνε κάθε στιγμή $Ν(Ν-1)/2=4290$ ανεξάρτητες ορατότητες και κατα τη διάρκεια μια πλήρους παρατήρησης, καθώς η γη περιστρέφεται, το κάθε ζευγάρι αντέννων διαγράφει ένα τμήμα έλλειψης στο επίπεδο $(u,v)$ και έτσι στο τέλος μια χαρτογραφικής παρατήρησης έχουν μετρηθεί εκατομμύρια ορατότητες ή αντίστοιχα συντελεστές {\en Fourier} της ειδικής έντασης της κοσμικής πηγής.  Τελικά η $I_{\nu}(l,m)$ δεν είναι τίποτε άλλο απο το μετασχηματισμό {\en Fourier} των {\en visibilities} $V(u,v)$. Ωστόσο είναι πρακτικά αδύνατο να συλλέξουμε όλες τις ορατότητες του επιπέδου παρατήρησης$\cdot$ με αποτέλεσμα η όποια πληροφορία δεν εμπίπτει στην κατανομή δειγματοληψίας των ορατοτήτων, $B(u,v)$, να μην λαμβάνεται ($V^{m}(u,v)=0$)

\begin{equation}
V^{m} (l,m) =  V(u,v)Β(u,v)
\end{equation}

όπου $Β(u,v)=1$ για μετρούμενη {\en visibility} ενώ $Β(u,v)=0$ για μη μετρούμενη. Σύμφωνα με το {\it Θεώρημα της Συνέλιξης} ο μετασχηματισμός {\en Fourier} του γινομένου δύο συναρτήσεων είναι η συνέλιξη των μετασχηματισμών {\en Fourier} της κάθε συνάρτησης\cite[{\en Appendix A.7}]{condon2016essential}! Τελικά ο μετασχηματισμος {\en Fourier} του γινομένου είναι: 

\begin{equation}\label{InverFourierDirtyIMage}
F[V^{m}(l,m)] =  F[V(u,v)]*F[Β(u,v)]
\end{equation}

όπου $F[V^{m}_v(l,m)]=I^{D}_{\nu}(l,m)$ ονομάζεται {\en Dirty Image} και τελικά:

\begin{equation}\label{InverFourierDirtyIMage2}
I^{D}_{\nu}(l,m) =  b(l,m)*I_{\nu}(l,m), \; \text{όπου $b(l,m)=F[B(u,v)]$}
\end{equation}

Η συνάρτηση $b(l,m)$ αποτελεί την συνθετική δέσμη του ραδιοσυμβολόμετρου {\en ({\it synthesized beam})}, συγκροτείται απο τον κεντρικό λοβό της δέσμης αλλά και απο μικρότερους λοβούς, οι οποίοι μπορούν να γίνουν και αρνητικοί, σε αντίθεση με τους αντίστοιχους {\it πλαγιολοβοούς} μονής αντέννας.
Ο κεντρικός λοβός προσεγγίζει γρήγορα μια Γκουσιανή δέσμη για σύγχρονα συμβολόμετρα όπου ο αριθμός των αντέννων, $N$, είναι μεγάλος. Είναι φανερό ότι το εύρος του κεντρικού λοβού, {\en (FWHM)}, καθορίζει το μέγεθος του και ταυτόχρονα την διακριτική ικανότητα του ραδιοσυμβολόμετρου. Το μέγεθος μιας Γκαουσιανής δέσμης δίνεται ως:

\begin{equation}\label{eq:GaussianBeam}
  \Delta\Omega_b = 1.133\times \theta^2, \text{όπου $\theta= FWHM$ σε $arcsec$}
\end{equation}

Στο σημείο αυτό φανερώνεται και το μεγάλο προτέρημα ενός ραδιοσυμβολόμετρου έναντι ενός μεμονομένου ραδιοτηλεσκοπίου$\cdot$ καθώς η διακριτική ικανότητα του δεύτερου δίνεται απο την σχέση:

\begin{equation}\label{eq:AnguralResolution}
\theta_{res} = k \frac{\lambda}{D}
\end{equation}   

όπου $k$ μια σταθερά, $\lambda$ το μήκος κύματος ανίχνευσης και $D$ η διάμετρος του <<πιάτου>> του ραδιοτηλεσκοπίου. Στην περίπτωση όμως ενός ραδιοσυμβολόμετρου η μεταβλητή $D$ στην παραπάνω σχέση αποτελεί {\it την απόσταση των ακραίων ραδιοτηλεσκοπίων της διάταξης} και ισοδυναμεί με τις μέγιστες τιμές $u_{max},v_{max}$ στο $(u,v) plane$. Για μια συγκεκριμένη συχνότητα, μεγάλες τιμές του $D$ συνεπάγονται μικρές τιμές του $\theta$ (δηλαδή μεγάλη διακριτική ικανότητα) άρα και σύμφωνα με την \eqref{eq:GaussianBeam} πολύ στενή δέσμη. Η χρήση μιας πολύ στενής δέσμης (μεγάλη διακριτική ικανότητα) μας επιτρέπει να ανιχνεύουμε λεπτομέρειες στην δομή των παρατηρούμενων αστρικών σωμάτων με μεγαλύτερη ευκρίνεια. Στον αντίποδα, με την χρήση ενός ραδιοσυμβολόμετρου εισέρχεται και ένας γηγενής περιορισμός. Πιο συγκεκριμένα οι ελάχιστες τιμές $u_{min}$ και $v_{min}$ καθορίζουν την μεγαλύτερη δυνατή κλίμακα χαρτογράφισης ενός αντικειμένου στο χώρο των χωρικών συχνοτήτων. Η κεντρική <<τρύπα>> στο $(u,v) plane$ μιας συγκεκριμένης συμβολομετρικής διάταξης, είναι εγγενής χαρακτηριστικό της διάταξης αυτής. Έτσι οποιαδήποτε λαμπρότητα του ουρανού περιγράφεται απο χωρικές συντεταγμένες $(u,v)$ που ξεπερνούν το όριο αυτό δεν συγκαταλέγονται στην απεικόνιση αλλα γίνονται {\en ``fillter-out``}. Το όριο αυτό δίνεται απο το μέγεθος $\vartheta_{MRS}$ ({\en Maximum Recoverable Scale}):

\begin{equation}\label{eq:MRS}
\vartheta_{MRS} \simeq \frac{0.6\lambda}{D_{min}}
\end{equation}
όπου το $D_{min}$ καθορίζεται απο τις $u_{min}$ και $v_{min}$.\\

Όπως φαίνεται απο την \eqref{eq:AnguralResolution} η διακριτική ικανότητα της συνθετικής ακτίνας εξαρτάται πέρα απο την συχνότητα στην οποία γίνεται η παρατήρηση και απο την διάταξη των ραδιοτηλεσκοπίων ({\en configuration}). Το {\en ALMA} μέσα απο τις διάφορες δυνατές διατάξεις των αντέννων του μπορεί να πετύχει και διαφορετικές τιμές διακριτικής ικανότητας ({\en angural resolution}), το οποίο μεταφράζεται σε διαφορετικά μεγέθη της συνθετικής ακτίνας. Παρακάτω δίνεται ο πίνακας με τις τιμές του {\en FWHM} για τις διάφορες συχνότητες στις οποίες θα γίνει η απεικόνιση του δίσκου:

\newpage

\begin{table}[h]
\centering
 \begin{tabular}{l | l | l | l }
  Συχνότητα {\en (gHz)} & {\en FHWM (arcsec)} & $\vartheta_{MRS}${\en(arcsec)} & {\en Configuration}\\
      \hline \hline
  230 & 0.0417 & 0.618 & {\en C-8} \\
  325 & 0.0278 & 0.412 & {\en C-8}\\
  460 & 0.0459 & 0.562 & {\en C-7}\\
  650 & 0.0471 & 0.632 & {\en C-6}\\
  870 & 0.0352 & 0.472 & {\en C-6}\\
 \end{tabular}
 \caption{Μέγιστη διακριτική ικανότητα του {\en ALMA} για τις διάφορες συχνότητες με  $\vartheta_{MRS}>0.408 arcsec$ (ακτινική διάμετρος του δίσκου)}\label{tab:ALMA}
\end{table} 


\begin{table}[h]
\centering
 \begin{tabular}{l | l | l | l }
  Συχνότητα {\en (gHz)} & {\en FHWM (arcsec)} & $\vartheta_{MRS}${\en(arcsec)} & {\en Configuration}\\
      \hline \hline
  230 & 0.0183 & 0.216 & {\en C-10} \\
  325 & 0.0122 & 0.144 & {\en C-10}\\
 \end{tabular}
 \caption{Μέγιστη διακριτική ικανότητα του {\en ALMA} για τις διάφορες συχνότητες με  $\vartheta_{MRS}<0.408 arcsec$(ακτινική διάμετρος του δίσκου)}\label{tab:ALMA2}
\end{table} 


Οι τιμές του πίνακα πάρθηκαν απο το {\en ALMA Cycle 8 Technical Handbook}\href{https://almascience.eso.org/documents-and-tools/cycle8/alma-technical-handbook}{{\en[ Chap.~7, Sect.~7.2 , ALMA]}} \\


 

